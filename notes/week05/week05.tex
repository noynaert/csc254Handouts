% !TEX program = xelatex
% Notes template for CSC 254
% The document is based on the exams class at https://math.mit.edu/~psh/exam/examdoc.pdf
% 
% This document uses the csc254.sty file

\documentclass[letterpaper,11pt]{exam}
\RequirePackage{amssymb, amsfonts, amsmath, graphicx, latexsym, verbatim, xspace, setspace}

% By default LaTeX uses large margins.  This doesn't work well on exams; problems
% end up in the "middle" of the page, reducing the amount of space for students
% to work on them.
\usepackage[margin=1in]{geometry}
\usepackage{xcolor}

% Here's where you edit the Class, Exam, Date, etc.
\newcommand{\class}{CSC 254}
\newcommand{\term}{NOTES}
\newcommand{\examnum}{Week 05, mostly chapter 03}
\newcommand{\videoheading}[1]{\Large\textbf{\textit{#1}}}


% For an exam, single spacing is most appropriate
\singlespacing
% \onehalfspacing
% \doublespacing

% For an exam, we generally want to turn off paragraph indentation
\parindent 0ex
\begin{document} 

% These commands set up the running header on the top of the exam pages
\pagestyle{head}
\firstpageheader{}{}{}
\runningheader{\class}{\examnum\ - Page \thepage\ of \numpages}{\includegraphics[width=1in]{javant}}
\runningheadrule

\begin{flushright}
\begin{tabular}{p{2.8in} r l}
\textbf{\class} & \textbf{Name (Print):} & \makebox[2in]{\hrulefill}\\
\textbf{\term} &&\\
\textbf{\examnum} &&\\
\end{tabular}\\
\end{flushright}
\rule[1ex]{\textwidth}{.1pt}


This assignment contains \numpages\ pages (including this page) and
\numquestions\ questions.  Check to see if any pages are missing.\\

These pages are indexed to the videos for this course.\\

\textbf{Grading}

This material comes mostly from Chapter 03 in the textbook.

\videoheading{Video 04.010 Booleans}


\begin{questions}
    \question What are the two possible values used by booleans?
    \question Create a boolean variable called \texttt{\textbf{hasWheels}} and set its initial value to false.
  \question Write the java symbol used for each operator:
  \begin{itemize}
    \item Less Than
    \item Less Than or Equal To
    \item Greater Than
    \item Greater Than Or Equal To
    \item Equal To
    \item Not Equal To
  \end{itemize}
  \question Suppose there is an integer variable called \texttt{\textbf{golfScore}}.  Create a boolean variable called \texttt{\textbf{beatPar}}.  Set beatPar to true if the golfScore is less than 70 and false if it was 70 or higher.  Do \textit{not} use an if statement or a conditional.
  \vspace{1.5cm}

\videoheading{Video 04.202 Truth Tables}

\begin{samepage}
  \question Assume the following variables are defined:
  \begin{verbatim}
      boolean isOldEnough = false;
      boolean isTallEnough = true;
      boolean isFastEnough = true; 
      boolean hasSeniority = false;
  \end{verbatim}
  Evaluate the following boolean expression.  (Is it true or false?)
  \begin{verbatim}
        isOldEnough || isTallEnough && isFastEnough || hasSeniority
  \end{verbatim}
  \textbf{Explain your answer in the space below}
  \vspace{2cm}
\end{samepage}
  

\begin{samepage}
\question Assume the following variables are defined:
\begin{verbatim}
    int age = 25;
    int pointsEarned = 25;
    int friendCount = 25;
\end{verbatim}
Evaluate the following boolean expression.  (Is it true or false?)
\begin{verbatim}
    age > 30 || pointsEarned > 20 && friendCount > 20
\end{verbatim}
\question Which of the following places the parenthesis so that it matches the default precedence?
\begin{verbatim}
  (age > 30 || pointsEarned > 20) && friendCount > 20
  age > 30 || (pointsEarned > 20 && friendCount > 20)
\end{verbatim}
\end{samepage}
\begin{minipage}{\textwidth}
  \question The sign at an amusement park ride says "You must be over 42 inches tall or be at least 8 years old and accompanied by an adult."  Assume there are defined boolean variables for \texttt{\textbf{over42}}, \texttt{\textbf{over8}}, and \texttt{\textbf{withAdult}}.  Write the boolean expression that evaluates whether the person may take the ride.
  \vspace{1cm}
  Complete the following truth table for the ride.\\
  \begin{tabular}{|c|c|c|c|}
    \hline
      over42 & over8 & withAdult & mayRide \\
      \hline
      0 & 0 & 0 & \\\hline
      0 & 0 & 1 & \\\hline
      0 & 1 & 0 & \\\hline
      0 & 1 & 1 & \\\hline
      1 & 0 & 0 & \\\hline
      1 & 0 & 1 & \\\hline
      1 & 1 & 0 & \\\hline
      1 & 1 & 1 & \\
    \hline
  \end{tabular}
\end{minipage}
\\
\\
\videoheading{Video 04.040 If Statements, Part 1}
\question  What are two syntax differences between Python and Java?
\begin{itemize}
  \item  
  \item  
\end{itemize}
\question When doesn't Java need \{ \} around the body of an if or else?
\vspace{1.cm}

\question Why is it usually a good idea to put in the curly braces, even when they are not needed.
\vspace{1.cm}
\question Assume the following two variables are declared.
\begin{verbatim}
  int age;
  double discount
\end{verbatim}
Write an if statement that assigns discount to 0.10 if age is at least 65 or if age is no more than 18.  Make sure you use proper indentation.
\vspace{3.cm}
\question Does every if statement need an else clause?  Explain your answer
\vspace{1.5cm}


\begin{minipage}{\textwidth}
  \videoheading{Video 04.030 Part 2, Two-handed hammering}\\
  Image attribution to https://freepix.com
  \question rewrite the following if statement so that it is not tacky.
  \begin{verbatim}
    if(age >= 21){
      isOldEnough = true;
    }else{
      isOldEnough = false;
    }
  \end{verbatim}
\end{minipage}
\begin{minipage}{\textwidth}
  \question rewrite the following if statement so that it is not tacky. (You only need to rewrite the tacky part.  In fact, you can just cross out the tacky part.)
  \begin{verbatim}
    if(isBirthday == true){
      System.out.println("Happy Birthday!");
    }else{
      System.out.println("Have a nice day.");
    }
  \end{verbatim}
\end{minipage}

\videoheading{Video 04.047 Danger Zone}

\question What is wrong with the following fragment of an if statement?
\begin{verbatim}
if(1.1 + 1.1 + 1.1 == 3.3)
\end{verbatim}
\question Is it OK to use == when working with Strings?  How are String variables different than type like int, double, and boolean?
\vspace{1cm}
\question Is it OK to use \textless and \textgreater operators when working with Strings?

\videoheading{Video 04.050 Switch Statement}

\question What data types may be used to control a switch statement (assuming the universal or classic Java version)
\question How is \texttt{\textbf{System.\textcolor{red}{err}.println("xyz");}} different than \texttt{\textbf{System.out.println("xyz");}}
\begin{minipage}{\textwidth}
\question A program prints the following menu on the screen.
\begin{verbatim}
    1. Broccoli Beef
    2. Hamburger Patty
    3. Bœuf  Haché 
    4. Fried Chicken
    5. Rice Bowl with Vegetables
\end{verbatim}
The program has read the user's choice into an integer called \texttt{\textbf{choice}}.  Your job is to write a switch statement that saves the user's choice as a variable called \texttt{\textbf{item}}. If the user picks any number other than 1 through 4, the item should be "Invalid Selection."  Also, Bœuf  Haché is just French for "Hamburger Patty."  So if the user picks either 2 or 3, it should be listed as Hamburger Patty. Use the "fall through" to do one assignment for both options 2 and 3.
\vspace{6cm}
\end{minipage}
\videoheading{Video 04.060 Conditional Operator}
\question You are writing the Java version of the son "99 Bottles of Beer on the Wall."  There is a variable for the number of bottles remaining.  You need the program to print the term "beers" when the number is plural or zero, and print "beer" when there is exactly 1 bottle.  Here is the if-statement version.  Write the same code as a single conditional.
\begin{verbatim}
  if(remaining == 1){
    System.out.printf("%d bottle of beer on the wall",remaining);
  }else{
    System.out.printf("%d bottles of beer on the wall", remaining);
  }
\end{verbatim}
\end{questions}
\begin{figure}[b]\label{end}
	\center
	\includegraphics[width=1in]{java}
\end{figure}
\end{document}
