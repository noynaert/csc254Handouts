% !TEX program = xelatex
% Notes template for CSC 254
% The document is based on the exams class at https://math.mit.edu/~psh/exam/examdoc.pdf
% 
% This document uses the csc254.sty file

\documentclass[letterpaper,11pt]{exam}
\RequirePackage{amssymb, amsfonts, amsmath, graphicx, latexsym, verbatim, xspace, setspace}

% By default LaTeX uses large margins.  This doesn't work well on exams; problems
% end up in the "middle" of the page, reducing the amount of space for students
% to work on them.
\usepackage[margin=1in]{geometry}
\usepackage{xcolor}

% Here's where you edit the Class, Exam, Date, etc.
\newcommand{\class}{CSC 254}
\newcommand{\term}{NOTES}
\newcommand{\examnum}{Week 09 Methods}
\newcommand{\videoheading}[1]{\Large\textbf{\textit{#1}}}


% For an exam, single spacing is most appropriate
\singlespacing
% \onehalfspacing
% \doublespacing

% For an exam, we generally want to turn off paragraph indentation
\parindent 0ex
\begin{document} 

% These commands set up the running header on the top of the exam pages
\pagestyle{head}
\firstpageheader{}{}{}
\runningheader{\class}{\examnum\ - Page \thepage\ of \numpages}{\includegraphics[width=1in]{javant}}
\runningheadrule

\begin{flushright}
\begin{tabular}{p{2.8in} r l}
\textbf{\class} & \textbf{Name (Print):} & \makebox[2in]{\hrulefill}\\
\textbf{\term} &&\\
\textbf{\examnum} &&\\
\end{tabular}\\
\end{flushright}
\rule[1ex]{\textwidth}{.1pt}


This assignment contains \numpages\ pages (including this page) and
\numquestions\ questions.  Check to see if any pages are missing.\\
\videoheading{This week's notes are based on the in-class session}
\begin{questions}

\question What is a method?
\vspace{2cm}

\question Why are all functions called methods in Java? (probably all are true, but which one is objectively accurate)
\begin{itemize}
  \item Because Java is weird 
  \item It's just a style thing
  \item In Java all functions must be in a method
  \item To make Java look different than C++
\end{itemize}

\question What are three advantages of using methods?
\begin{itemize}
  \item ~
  \item ~
  \item ~
\end{itemize}

\question How is a static method called?
\vspace{1cm}
\question Give an example of how \texttt{\textbf{parseInt}} in class \texttt{\textbf{Integer}} would be called on a string called \texttt{\textbf{number}}.
\question How is a non-static method called?
\question Give an example of how \texttt{\textbf{length}} in class \texttt{\textbf{String}} would be called on a string called \texttt{\textbf{number}}.
\question When a static method is called \textit{fom within its own class} how can it be called?  
\vspace{1cm}
\question Does the method alluded to in the previous question also apply to non-static methods?
\vspace{1cm}
\question What is the syntax for creating a method?
\vspace{1cm}
\question Write a method named \texttt{\textbf{printBar}}.  It takes no arguments and has no return value.  It only prints \texttt{\textbf{===============}}.
\vspace{4cm}
\question Write a method named \texttt{\textbf{greeting}}.  It takes a string argument and has no return value.  It only prints \texttt{\textbf{===============}}.
\vspace{4cm}
\question Write a method named \texttt{\textbf{square}}.  It takes a double as an argument and returns a double.  It only prints \texttt{\textbf{===============}}.
\vspace{4cm}
\question What is the difference between a parameter and an argument?

hint: It might help if you think of "parameter" as "formal paraeter." 
\vspace{1cm}

\question What is the difference between public and private?
\question So far, all of our course we have only had a single class in our programs.  Does the difference between public and private matter right now?
\vspace{1cm}

\question What is the return type of a method that does not return a value?
\question How is a method that returns a value called differently than a function that does not return a value?
\vspace{1cm}
\question Does a method that calls a function have to use the return value, or may it be used like a regular statement if we want to ignore the return value?  (Try it and find out if you need to)

\question What is "Call by value?"  Why is it also called "Call by copy?"  

\question What is a "Call by reference?"  

\question "Java only has call by value"
\begin{itemize}
  \item  Absolutely True with no fine print
  \item  False
  \item Technically \&\& Absolutely True, but with a huge loophole in the fine print.
\end{itemize}


\question What is the "method signature?"
\vspace{1cm}
\question Are the formal parameters part of the method signature?
\question Is the return type part of the method signature

\question What is "method overloading?"
\vspace{1cm}
\question What is the relationship between the method signature and method overloading?
\vspace{1cm}
\question Can a method be overloaded if the only difference is the return type?  Explain why (which should be pretty obvious after the last three questions)
\vspace{1cm}
\question How does scope apply to methods?
\vspace{1cm}

\question What are "local variables?"
\vspace{1cm}
\question Can two different methods have the same names as their formal parameters?
\question Can a method have a parameter with the same name as one used in the main() method?
\question Hey, wait a minute.  Is \texttt{\textbf{main(String[] args)}} a method?
\end{questions}

\end{document}
