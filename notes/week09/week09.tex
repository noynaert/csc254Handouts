% !TEX program = xelatex
% Notes template for CSC 254
% The document is based on the exams class at https://math.mit.edu/~psh/exam/examdoc.pdf
% 
% This document uses the csc254.sty file

\documentclass[letterpaper,11pt]{exam}
\RequirePackage{amssymb, amsfonts, amsmath, graphicx, latexsym, verbatim, xspace, setspace}

% By default LaTeX uses large margins.  This doesn't work well on exams; problems
% end up in the "middle" of the page, reducing the amount of space for students
% to work on them.
\usepackage[margin=1in]{geometry}
\usepackage{xcolor}

% Here's where you edit the Class, Exam, Date, etc.
\newcommand{\class}{CSC 254}
\newcommand{\term}{NOTES}
\newcommand{\examnum}{Week 09 Methods}
\newcommand{\videoheading}[1]{\Large\textbf{\textit{#1}}}


% For an exam, single spacing is most appropriate
\singlespacing
% \onehalfspacing
% \doublespacing

% For an exam, we generally want to turn off paragraph indentation
\parindent 0ex
\begin{document} 

% These commands set up the running header on the top of the exam pages
\pagestyle{head}
\firstpageheader{}{}{}
\runningheader{\class}{\examnum\ - Page \thepage\ of \numpages}{\includegraphics[width=1in]{javant}}
\runningheadrule

\begin{flushright}
\begin{tabular}{p{2.8in} r l}
\textbf{\class} & \textbf{Name (Print):} & \makebox[2in]{\hrulefill}\\
\textbf{\term} &&\\
\textbf{\examnum} &&\\
\end{tabular}\\
\end{flushright}
\rule[1ex]{\textwidth}{.1pt}


This assignment contains \numpages\ pages (including this page) and
\numquestions\ questions.  Check to see if any pages are missing.\\
\videoheading{This week's notes are based on the in-class session}
\begin{questions}

\question What is a method?
\vspace{2cm}

\question Why are all functions called methods in Java? (probably all are true, but which one is objectively accurate)
\begin{itemize}
  \item Because Java is weird 
  \item It's just a style thing
  \item In Java all functions must be in a method
  \item To make Java look different than C++
\end{itemize}

\question How is a static method called?
\vspace{1cm}
\question Give an example of how \texttt{\textbf{parseInt}} in class \texttt{\textbf{Integer}} would be called on a string called \texttt{\textbf{number}}.
\question How is a non-static method called?
\question Give an example of how \texttt{\textbf{length}} in class \texttt{\textbf{String}} would be called on a string called \texttt{\textbf{number}}.
\question When a static method is called \textit{fom within its own class} how can it be called?  
\vspace{1cm}






\end{questions}

\end{document}
