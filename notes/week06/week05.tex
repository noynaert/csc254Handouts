% !TEX program = xelatex
% Notes template for CSC 254
% The document is based on the exams class at https://math.mit.edu/~psh/exam/examdoc.pdf
% 
% This document uses the csc254.sty file

\documentclass[letterpaper,11pt]{exam}
\RequirePackage{amssymb, amsfonts, amsmath, graphicx, latexsym, verbatim, xspace, setspace}

% By default LaTeX uses large margins.  This doesn't work well on exams; problems
% end up in the "middle" of the page, reducing the amount of space for students
% to work on them.
\usepackage[margin=1in]{geometry}
\usepackage{xcolor}

% Here's where you edit the Class, Exam, Date, etc.
\newcommand{\class}{CSC 254}
\newcommand{\term}{NOTES}
\newcommand{\examnum}{Week 06}
\newcommand{\videoheading}[1]{\Large\textbf{\textit{#1}}}


% For an exam, single spacing is most appropriate
\singlespacing
% \onehalfspacing
% \doublespacing

% For an exam, we generally want to turn off paragraph indentation
\parindent 0ex
\begin{document} 

% These commands set up the running header on the top of the exam pages
\pagestyle{head}
\firstpageheader{}{}{}
\runningheader{\class}{\examnum\ - Page \thepage\ of \numpages}{\includegraphics[width=1in]{javant}}
\runningheadrule

\begin{flushright}
\begin{tabular}{p{2.8in} r l}
\textbf{\class} & \textbf{Name (Print):} & \makebox[2in]{\hrulefill}\\
\textbf{\term} &&\\
\textbf{\examnum} &&\\
\end{tabular}\\
\end{flushright}
\rule[1ex]{\textwidth}{.1pt}


This assignment contains \numpages\ pages (including this page) and
\numquestions\ questions.  Check to see if any pages are missing.\\

These pages are indexed to the videos for this course.\\

\textbf{Grading}

\videoheading{Video 06.010 The Math API}


\begin{questions}
    \question What is the API?
    \vspace*{1cm}

    \question Why is Math capitalized?

    \question What are the two fields in the Math class?  Why are they all caps?
    \vspace{1cm}

    \question In practice, what does \texttt{\textbf{static}} mean?
    \vspace{.5cm}

    \question What are "methods?"
    \vspace{.5cm}

    Comment:  (You don't have to answer this.) In the video I mention that Java programmers have to look up methods.  One thing I did not mention is that the Java guidelines often mean that the programmer can just guess what the method name will be.  Once you understand the pattern of method names, guessing methods is pretty easy.

  \question What is the method to calculate an absolute value?  
  \question What is the return type of abs() (Note that this is a tricky question)
  \vspace{.75cm}
  

\videoheading{Video 04.020 Random Numbers}

\question What are some applications that use random numbers?
\begin{itemize}
  \item ~
  \item ~
\end{itemize}
\question What are "pseudorandom" numbers?

\question What method do you call to generate a random number in Java?

\question What is the return type of Math.random()?

\question What is the smallest number that Math.random() can generate?
\question What is the largest number that Math.random() can generate?  (This is kind of hard to answer.)
\vspace{.5cm}

\question Which of the following numbers could be generated by Math.random()?
\begin{itemize}
  \item  0.20623453066155573
  \item  0.09177047425069951
  \item  0.00000000000000000
  \item 1.000000000000000000
  \item 7.324442222444424243
  \item  0.99999999999999999
\end{itemize}
\question Why is it handy to think of the output of Math.random() as a percentage?
\question The following conditional statement simulates a coin toss.  Why is this coin toss not "fair?"  Rewrite the conditional statement so that it is has an equal chance of giving heads or tails.
\begin{verbatim}
  String result = (Math.random() < 0.6)?"Heads":"Tails";
\end{verbatim}
\vspace{1cm}
\question Write a Java expression that generates a random number from 0 through 99.  \textit{Note that (int)(Math.random()*99) is not the correct answer.}
\vspace{1cm}
\question Write a statement that generates a random number from 1 through 100.
\vspace{1cm}

\begin{samepage}
Summary of the Math methods you need to know:
\begin{itemize}
  \item \texttt{\textbf{abs}}  Note that the return type varies based on the type of the argument.
\end{itemize}
\end{samepage}
 
\end{questions}
\begin{figure}[b]\label{end}
	\center
	\includegraphics[width=1in]{java}
\end{figure}
\end{document}
