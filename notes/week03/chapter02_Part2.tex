% !TEX program = xelatex
% Notes template for CSC 254
% The document is based on the exams class at https://math.mit.edu/~psh/exam/examdoc.pdf
% 
% This document uses the csc254.sty file

\documentclass[letterpaper,11pt]{exam}
\RequirePackage{amssymb, amsfonts, amsmath, graphicx, latexsym, verbatim, xspace, setspace}

% By default LaTeX uses large margins.  This doesn't work well on exams; problems
% end up in the "middle" of the page, reducing the amount of space for students
% to work on them.
\usepackage[margin=1in]{geometry}


% Here's where you edit the Class, Exam, Date, etc.
\newcommand{\class}{CSC 254}
\newcommand{\term}{NOTES}
\newcommand{\examnum}{Week 03}
\newcommand{\videoheading}[1]{\Large\textbf{\textit{#1}}}


% For an exam, single spacing is most appropriate
\singlespacing
% \onehalfspacing
% \doublespacing

% For an exam, we generally want to turn off paragraph indentation
\parindent 0ex
\begin{document} 

% These commands set up the running header on the top of the exam pages
\pagestyle{head}
\firstpageheader{}{}{}
\runningheader{\class}{\examnum\ - Page \thepage\ of \numpages}{\includegraphics[width=1in]{javant}}
\runningheadrule

\begin{flushright}
\begin{tabular}{p{2.8in} r l}
\textbf{\class} & \textbf{Name (Print):} & \makebox[2in]{\hrulefill}\\
\textbf{\term} &&\\
\textbf{\examnum} &&\\
\end{tabular}\\
\end{flushright}
\rule[1ex]{\textwidth}{.1pt}


This assignment contains \numpages\ pages (including this page) and
\numquestions\ questions.  Check to see if any pages are missing.\\

These pages are indexed to the videos for this course.\\

\textbf{Grading}

Your answers must be hand-written on printed copies of the notes.  
\begin{itemize}
 
    \item This document will only be "spot checked."  Not all answers may be graded.
    \item \textit{Don't skip questions!} If you don't know an answer, at least give it a try.  Or explain what you don't understand.  Put a big star or arrow around the points you don't understand and ask about them in class.
    \item Keep your copy.  When I make up the test I will go down the notes page and draw inspiration from these questions when making up the exam. These notes pages will be good study guides.
    \item You may write any lingering questions or muddy points at the end of the document.  Then ask in class
    \item Scan this document into a .pdf file and turn it in.  If you don't have access to a regular scanner then install a "CamScanner" app on your phone or tablet.  They work pretty well.  Photographs will not be accepted.  You must scan to a .pdf document.
\end{itemize}

\videoheading{Video 02.035 Part 1}


\begin{questions}
    \question What import do you have to do to use the Scanner class?
    \question What is the URL for the java API?  I used version 17 in the video.  Give the URL for the version of Java that you are using.
    \question What constructor would we use if we want to read from a file?
    \question What constructor will we use to read from a file?
    \question What is the difference between a "method" and a "function?"
    \begin{samepage}
    \question What data type is returned by each of the following?  These are not answered at this point in the video.  Use the API website to figure out the types that are returned.
    \begin{itemize}
      \item nextDouble() returns a \makebox[2in]{\hrulefill}
      \item nextInt() returns a \makebox[2in]{\hrulefill}
      \item next() returns a \makebox[2in]{\hrulefill}
      \item nextLine() returns a \makebox[2in]{\hrulefill}
    \end{itemize}
    \end{samepage}

\question Do you have to use the id "keyboard" when reading from the keyboard?  Explain your answer.
\vspace{1cm}
\question What is the name of the InputStream used when reading from the keyboard.  (Hint:  The answer is not System.out)

\question What command would close a stream.  Assume the Scanner is called "input."
\begin{samepage}
\question what is the line of code that would be needed to do the following task?
\begin{verbatim}
  Read a double from the keyboard into a variable named \texttt{\textfb{x}};
\end{verbatim}
\end{samepage}
\begin{samepage}
\question Given the above statement that reads a variable into x, what would happen if the user typed the following input.  
\begin{verbatim}
    7.143  0202
\end{verbatim}
\end{samepage}
\question What happens if a the user hits the Enter key or blank space when the Scanner is trying to read a number from the keyboard?
\question What happens if the user types a number with two decimal points.  Perhaps something like 12.34.56 when the Scanner is trying to read a double (The video never does this, but you can probably guess.  Also, you should be running a playground-type program running in IntelliJ so you should be able to confirm your answer.)
\question An exception has occurred.  
\begin{samepage}
\question Suppose Intellij produces the following error.  What type of exception occured?  What line number did the exception happen in?
\begin{verbatim}
  Exception in thread "main" java.util.InputMismatchException
	at java.base/java.util.Scanner.throwFor(Scanner.java:939)
	at java.base/java.util.Scanner.next(Scanner.java:1594)
	at java.base/java.util.Scanner.nextInt(Scanner.java:2258)
	at java.base/java.util.Scanner.nextInt(Scanner.java:2212)
	at App.main(App.java:16)
\end{verbatim}
  What is the name of the exception?\\
  In your program, what line number generated the error?
\end{samepage}
\question What is the difference between next() and nextLine()?
\vspace{1cm}
\question What happens if you use next() to read from the keyboard, and the user types a number?
\question Write a complete program that prompts a user for an integer and reads it from the keyboard.  It should also read a string from the keyboard. Include the necessary import  You do NOT need to include the /**javadoc*/ comment.
\vspace{2in}

\begin{samepage}
\videoheading{Video 02.037 Other ways to Import}
\question Write an example of the three different ways to import the Scanner class.
\begin{enumerate}
  \item
  \item
  \item
\end{enumerate}
\end{samepage}
\question What does IntelliJ do when you tell it to "Optimize Imports?"
\vspace{1cm}

\videoheading{Video 02.040 Expressions}

\question What is an expression?
\begin{samepage}
\question List things that can be an expression
\begin{itemize}
  \item 
  \item 
  \item 
  \item 
\end{itemize}
\end{samepage}
\question Can an expression contain an expression?  \textbf{Give an example}


\videoheading{Video 02.040 Part 02}
\begin{samepage}
\question Evaluate each of the following expressions using the Java "order of operations."  Be sure to pay attention to data types.
\begin{itemize}
  \item$1+2*3$
  \item$10/2$
  \item $5/2$
  \item $5./2.$
  \item $5/2.$
  \item $25/2*100$
  \item $25./50*100$
  \item $25/50.*100$
  \item $25/50.*100$
  \item $25/50*100.$
\end{itemize}
\end{samepage}
\question Explain why $1/2 * 5. * 2$ gives 0. as an answer, but $5. * 2 * 1/2$ gives 5.
\vspace{1cm}
\\\\
\videoheading{Video 02.040 Part 03 Math Functions}
\question What is the return type of most Math functions?
\question Write the Math function to return the square root of 99.
\\
\question Write the Math function to return the absolute value of -99.
\\
\question What does Math.random() return?
\question What is the smallest value that Math.random() return?
\question Math.random() can produce a number up to \hrulefill

\question What is the exponentiation operator in Java?  (this is a trick question, because there isn't one.  Oops.  I just answered the question.  Never mind.  Move along.)
\question Write the expression that would return $4.4^2$ in Java.  Do it two different ways.
\\
\question Write the expression that would return $3.4^{-2.7}$ in Java.
\\
\question Write the expression that would return $2^{3^4}$ in Java.  (That is, 2 raised to the 3rd raised to the 4th)
\\
\videoheading{Video 02.040 Part 04}

\question What java function would return the length of a string? \makebox[2cm]{\hrulefill}
\question What does the java trim() function do?
\\
\begin{samepage}
\question What is the \textit{return type} of each of the following Java functions?
\begin{itemize}
  \item len() returns a \makebox[2cm]{\hrulefill}
  \item trim() returns a \makebox[2cm]{\hrulefill}
  \item equals() returns a \makebox[2cm]{\hrulefill}
  \item contains() returns a \makebox[2cm]{\hrulefill}
\end{itemize}
\end{samepage}
\question Use the API to locate the "index of" a character in a string.  What is the return type of this function?
\\
\question What is the operator used to "concatenate" Strings? \makebox[2cm]{\hrulefill}
\question What happens when you "concatenate" a String and a number?
\\
\question What would be printed by the following code?
\begin{verbatim}
    System.out.println("3" + "3");
\end{verbatim}









\end{questions}
\begin{figure}[b]\label{end}
	\center
	\includegraphics[width=1in]{java}
\end{figure}
\end{document}
