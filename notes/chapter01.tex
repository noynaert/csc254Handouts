% Exam Template for UMTYMP and Math Department courses
%
% Using Philip Hirschhorn's exam.cls: http://www-math.mit.edu/~psh/#ExamCls
%
% run pdflatex on a finished exam at least three times to do the grading table on front page.
%
%%%%%%%%%%%%%%%%%%%%%%%%%%%%%%%%%%%%%%%%%%%%%%%%%%%%%%%%%%%%%%%%%%%%%%%%%%%%%%%%%%%%%%%%%%%%%%

% These lines can probably stay unchanged, although you can remove the last
% two packages if you're not making pictures with tikz.
\documentclass[11pt]{exam}
\RequirePackage{amssymb, amsfonts, amsmath, latexsym, verbatim, xspace, setspace}
\RequirePackage{tikz, pgflibraryplotmarks}

% By default LaTeX uses large margins.  This doesn't work well on exams; problems
% end up in the "middle" of the page, reducing the amount of space for students
% to work on them.
\usepackage[margin=1in]{geometry}


% Here's where you edit the Class, Exam, Date, etc.
\newcommand{\class}{CSC 254}
\newcommand{\term}{NOTES}
\newcommand{\examnum}{Chapter 01}


% For an exam, single spacing is most appropriate
\singlespacing
% \onehalfspacing
% \doublespacing

% For an exam, we generally want to turn off paragraph indentation
\parindent 0ex

\begin{document} 

% These commands set up the running header on the top of the exam pages
\pagestyle{head}
\firstpageheader{}{}{}
\runningheader{\class}{\examnum\ - Page \thepage\ of \numpages}
\runningheadrule

\begin{flushright}
\begin{tabular}{p{2.8in} r l}
\textbf{\class} & \textbf{Name (Print):} & \makebox[2in]{\hrulefill}\\
\textbf{\term} &&\\
\textbf{\examnum} &&\\
\end{tabular}\\
\end{flushright}
\rule[1ex]{\textwidth}{.1pt}


This assignment contains \numpages\ pages (including this page) and
\numquestions\ problems.  Check to see if any pages are missing.\\

These pages are indexed to the videos for this course.\\

\textbf{Grading}

Your answers must be hand-written on printed copies of the notes.  
\begin{itemize}
    \item I will not accept typed answers
    \item I will not accept answers written on notebook paper.
    \item This document will only be "spot checked."  Not all answers may be graded.
    \item \textit{Don't skip questions!} If you don't know an answer, at least give it a try.  Or explain what you don't understand.  Put a big star or arrow around the points you don't understand and ask about them in class.
    \item Keep your copy.  When I make up the test I will go down the notes page and draw inspiration from these questions when making up the exam. These notes pages will be good study guides.
    \item You may write any lingering questions or muddy points at the end of the document.  Then ask in class
    \item Scan this document into a .pdf file and turn it in.  If you don't have access to a regular scanner then install a "CamScanner" app on your phone or tablet.  They work pretty well.  Photographs will not be accepted.  You must scan to a .pdf document.
\end{itemize}


%\newpage % End of cover page

%%%%%%%%%%%%%%%%%%%%%%%%%%%%%%%%%%%%%%%%%%%%%%%%%%%%%%%%%%%%%%%%%%%%%%%%%%%%%%%%%%%%%
%
% See http://www-math.mit.edu/~psh/#ExamCls for full documentation, but the questions
% below give an idea of how to write questions [with parts] and have the points
% tracked automatically on the cover page.
%
%
%%%%%%%%%%%%%%%%%%%%%%%%%%%%%%%%%%%%%%%%%%%%%%%%%%%%%%%%%%%%%%%%%%%%%%%%%%%%%%%%%%%%%

\textit{\textbf{Video 01.010 Course Overview}}

\begin{questions}

% Basic question 

\question What are some good things about Python?
\vspace{2cm}
\question What are some limitations or problems with Java?
\vspace{2cm}

\vspace{2cm}
\question What are some limitations or problems with Java?
\vspace{2cm}

% Question with parts
%\newpage
%\question Consider the function $f(x)=x^2$.
%\begin{parts}
%\part[5] Find $f'(x)$ using the limit definition of derivative.
%\vfill
%\part[5] Find the line tangent to the graph of $y=f(x)$ at the point where $x=2$.
%\vfill
%\end{parts}

% If you want the total number of points for a question displayed at the top,
% as well as the number of points for each part, then you must turn off the point-counter
% or they will be double counted.
%\question[10] Consider the function $f(x)=x^3$.
%\noaddpoints % If you remove this line, the grading table will show 20 points for this problem.
%\begin{parts}
%\part[5] Find $f'(x)$ using the limit definition of derivative.
%\vspace{4.5in}
%\part[5] Find the line tangent to the graph of $y=f(x)$ at the point where $x=2$.
%\end{parts}


\textit{\textbf{Video 01.020 JDK \& JRE}}

\question What's up?
\end{questions}
\end{document}
