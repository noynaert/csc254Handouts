% !TEX program = xelatex
% Notes template for CSC 254
% The document is based on the exams class at https://math.mit.edu/~psh/exam/examdoc.pdf
% 
% This document uses the csc254.sty file

\documentclass[letterpaper,11pt]{exam}
\RequirePackage{amssymb, amsfonts, amsmath, graphicx, latexsym, verbatim, xspace, setspace}

% By default LaTeX uses large margins.  This doesn't work well on exams; problems
% end up in the "middle" of the page, reducing the amount of space for students
% to work on them.
\usepackage[margin=1in]{geometry}
\usepackage{xcolor}

% Here's where you edit the Class, Exam, Date, etc.
\newcommand{\class}{CSC 254}
\newcommand{\term}{NOTES}
\newcommand{\examnum}{Week 07}
\newcommand{\videoheading}[1]{\Large\textbf{\textit{#1}}}


% For an exam, single spacing is most appropriate
\singlespacing
% \onehalfspacing
% \doublespacing

% For an exam, we generally want to turn off paragraph indentation
\parindent 0ex
\begin{document} 

% These commands set up the running header on the top of the exam pages
\pagestyle{head}
\firstpageheader{}{}{}
\runningheader{\class}{\examnum\ - Page \thepage\ of \numpages}{\includegraphics[width=1in]{javant}}
\runningheadrule

\begin{flushright}
\begin{tabular}{p{2.8in} r l}
\textbf{\class} & \textbf{Name (Print):} & \makebox[2in]{\hrulefill}\\
\textbf{\term} &&\\
\textbf{\examnum} &&\\
\end{tabular}\\
\end{flushright}
\rule[1ex]{\textwidth}{.1pt}


This assignment contains \numpages\ pages (including this page) and
\numquestions\ questions.  Check to see if any pages are missing.\\

These pages are indexed to the videos for this course.\\

\textbf{Grading}

\videoheading{Video 06.010 The Math API}


\begin{questions}
    \question What is the API?
    \vspace*{1cm}

    \question Why is Math capitalized?

    \question What are the two fields in the Math class?  Why are they all caps?
    \vspace{1cm}

    \question In practice, what does \texttt{\textbf{static}} mean?
    \vspace{.5cm}

    \question What are "methods?"
    \vspace{.5cm}

    Comment:  (You don't have to answer this.) In the video I mention that Java programmers have to look up methods.  One thing I did not mention is that the Java guidelines often mean that the programmer can just guess what the method name will be.  Once you understand the pattern of method names, guessing methods is pretty easy.

  \question What is the method to calculate an absolute value?  
  \question What is the return type of abs() (Note that this is a tricky question)
  \vspace{.75cm}
  

\videoheading{Video 04.020 Random Numbers}

\question What are some applications that use random numbers?
\begin{itemize}
  \item ~
  \item ~
\end{itemize}
\question What are "pseudorandom" numbers?

\question What method do you call to generate a random number in Java?

\question What is the return type of Math.random()?

\question What is the smallest number that Math.random() can generate?
\question What is the largest number that Math.random() can generate?  (This is kind of hard to answer.)
\vspace{.5cm}

\question Which of the following numbers could be generated by Math.random()?
\begin{itemize}
  \item  0.20623453066155573
  \item  0.09177047425069951
  \item  0.00000000000000000
  \item 1.000000000000000000
  \item 7.324442222444424243
  \item  0.99999999999999999
\end{itemize}
\question Why is it handy to think of the output of Math.random() as a percentage?
\question The following conditional statement simulates a coin toss.  Why is this coin toss not "fair?"  Rewrite the conditional statement so that it is has an equal chance of giving heads or tails.
\begin{verbatim}
  String result = (Math.random() < 0.6)?"Heads":"Tails";
\end{verbatim}
\vspace{1cm}
\question Write a Java expression that generates a random number from 0 through 99.\\  \textit{Note that (int)(Math.random()*99) is not the correct answer.}
\vspace{1cm}
\question Write a statement that generates a random number from 1 through 100.
\vspace{1cm}

\videoheading{06.030 Math Methods}
\question Write a math statement that would calculate the absolute value of -33.2 and store the value in a variable.  Make sure the variable is of the correct type.

\color{cyan}I am providing the answer to this question so you can see the pattern.
\color{blue}
\begin{verbatim}
           double x = Math.abs(-33.2);
\end{verbatim}
\color{black}

\question Write a math statement that would calculate ${5.0}^{8.3}$ and store the value in a variable.  Make sure the variable is of the correct type.
\vspace{1cm}

\question Write a math statement that would calculate $\pi^e$ and store the value in a variable.  Make sure the variable is of the correct type.
\vspace{1cm}

\question Write a math statement that would calculate $\sqrt{y}$ and store the value in a variable.  Make sure the variable is of the correct type.
\vspace{1cm}

\question Write a math statement that would calculate $\sqrt[5]{\pi}$ and store the value in a variable.  Make sure the variable is of the correct type.
\vspace{1cm}

\question Write a math statement that would calculate the integer value that "rounds down" by ignoring the fractional part of a real number.  Do not convert to type int.  Store the value in a double.  
\vspace{1cm}

\question Write a math statement that would calculate the integer value that would round a real number to the nearest integer.  Store the value in an int (be careful!).  
\vspace{1cm}


\question Write a math statement that would calculate the integer value that "rounds down" by ignoring the fractional part of a real number.  Do not convert to type int.  Store the value in a double.  
\vspace{1cm}


\question Write a math statement that would calculate log(x) and store the value in a variable.  Make sure the variable is of the correct type.
\vspace{1cm}

\question Write a math statement that would calculate $\log_{10}{500}$ and store the value in a variable.  Make sure the variable is of the correct type.
\vspace{1cm}

\question Math.round(double) returns a long.  Math.round(float) returns an int.  Why would Math.round(double) need to return a long and not an int?
\vspace{1cm}

\question What would be the difference between \texttt{\textbf{'a'}} and \texttt{\textbf{"a"}}?
\vspace{1cm}

\question Why would the following statement be an error?  Rewrite it so that it is correct.
\begin{verbatim}
       int number = '5';
\end{verbatim}

\videoheading{06.035 ASCII and Unicode}

\question What two Java data types would use ASCII and Unicode?

\question What is the problem with ASCII codes?

\question Are ASCII codes part of Unicode?

\question Look up the following values in an ASCII table.  What is the ASCII value of each symbol?
\begin{itemize}
  \item 'A'
  \item 'a'
  \item '0' (zero)
  \item ' ' (blank space)
  \item '?'
  \item '~'
\end{itemize}
\question In the video I said that 16-bit Unicode can have 256 alphabets and each alphabet can have up to 256 characters.  I never explained why there were 256 alphabets or why each alphabet could have 256 characters.  If Unicode uses 16 bits, then why is the restriction based on 256?
\vspace{1cm}
\question In both ASCII and Unicode, each symbol is represented as an \rule{2cm}{0.15mm}.

\question Look up the unicode value of the "yen" ¥ symbol.  Write the yen symbol in \u notation.  
\vspace{.5cm}

Note:  In the rest of the video I had fun with emoji characters.  Watch it if you wish.  

\videoheading{06.040 char data type}

\question How is a char different than a string?
\vspace{1cm}

\question String literals are marked by "double quote" marks.  How are char literals marked?

\question What is the escape sequence for each of the following?
\begin{itemize}
  \item new line
  \item The \ symbol
  \item quote mark
  \item tab
\end{itemize}
\question State whether each statement is true or false.  You might want to these in a little program or jshell.
\begin{itemize}
  \item 'Z' \textless 'z'
  \item 'Z' \textless 's'
  \item 'z' \textless '6' 
  \item ' ' \textless '6'
  \item 'Z' \textless '6'
  \item 'Z' == 'z'
\end{itemize}

\question Write a statement that would determine if a character called \texttt{\textbf{ch}} is a digit and store the value in a variable.  Make sure the variable is of the correct type.
\vspace{1cm}
\question Write a statement that would determine if a character called \texttt{\textbf{ch}} is a letter of the alphabet and store the value in a variable.  Make sure the variable is of the correct type.
\vspace{1cm}
\question Write a statement that would determine if a character called \texttt{\textbf{ch}} is upper Case and store the value in a variable.  Make sure the variable is of the correct type.
\vspace{1cm}
\question Write a statement that would convert a character called \texttt{\textbf{ch}} and store the value in a variable.  Make sure the variable is of the correct type.
\vspace{1cm}

\videoheading{06.050 String Class}
\question Write a statement that would determine the length of a String called \texttt{\textbf{s}} and store the value in a variable.  Make sure the variable is of the correct type.
\vspace{1cm}
\question Write a statement that would determine the character at position 0 of a String called \texttt{\textbf{s}} and store the value in a variable.  Make sure the variable is of the correct type.
\vspace{1cm}
\question Write a statement that would convert a String called \texttt{\textbf{s}} to all upper case and store the value in a variable.  Make sure the variable is of the correct type.
\vspace{1cm}
\question Assume there are two integer variables called \texttt{\textbf{apples}} and \texttt{\textbf{bananas}}.  Write a statement that uses the String.format statement to create a String that contains a sentence like "There are 4 apples and 3 bananas." and store it in a String variable.
\vspace{1cm}

\videoheading{The following 4 questions are not in the video.  See if you can figure them out.  The second, third, and forth  ones are a bit challenging, but I have confidence you can figure them out.}

\question There is a string \texttt{\textbf{s}}.  Create a statement that stores the \textit{first} character of the string in a variable.  You may assume the string has at least 1 character.
\vspace{1cm}

\question There is a string \texttt{\textbf{s}}.  Create a statement that stores the \textit{last} character of the string in a variable.  You may assume the string has at least 1 character.
\vspace{1cm}

\question There is a string \texttt{\textbf{s}}.  Create a statement that converts the first character (and only the first character) of the string in a to upper case.  You may assume the string has at least 1 character.
\vspace{1cm}

\question There is a string \texttt{\textbf{s}}.  The string may be empty (length of 0) or it may have characters.  If the string has at least one character, then make the entire string lower case, then convert the first letter of the string to upper case.  Store the new string in a variable \texttt{\textbf{t}} that is declare outside the if statement.  If the string has length zero, then set \texttt{\textbf{t}} to an empty string.
\vspace{4cm}


\videoheading{06.070 String Comparison, Part 1}

For the remaining questions, assume there are two string variables named \texttt{\textbf{first}} and \texttt{\textbf{second}}

\question Write a boolean statement that would set a variable \texttt{\textbf{isSame}} if \texttt{\textbf{first}} and \texttt{\textbf{second}} are equal (cases must match exactly).
\vspace{1cm}

\question Write a boolean statement that would set a variable \texttt{\textbf{isSame}} if \texttt{\textbf{first}} and \texttt{\textbf{second}} are equal (ingore case).
\vspace{1cm}

\question Write a boolean statement that would set a variable \texttt{\textbf{isDifferent}} if \texttt{\textbf{first}} and \texttt{\textbf{second}} are not equal (cases must match exactly).
\vspace{1cm}

\videoheading{06.070 String Comparison, Part 2}
\question Write an if statement that would print "The first word is less than the second" if \texttt{\textbf{first}} is alphabetically before \texttt{\textbf{second}}.  Don't do anything if the two values are the second, or if the first is greater than the second. (cases must match exactly).  Note that you don't have to write an else clause.
\vspace{1.5cm}
\question Write an if statement that would print "The first word is less than the second" if \texttt{\textbf{first}} is alphabetically before \texttt{\textbf{second}}.  Don't do anything if the two values are the second, or if the first is greater than the second. (ignore case).  Note that you don't have to write an else clause.
\vspace{1.5cm}



\begin{samepage}
Summary of the Math methods you need to know. (You don't have to write anything here.  It is just a list for your reference when studying)
\begin{itemize}
  \item \texttt{\textbf{abs}}  Note that the return type varies based on the type of the argument.
  \item ceil
  \item floor
  \item sqrt
  \item round
  \item pow
  \item log
  \item log10
\end{itemize}


Summary of char methods you need to know. (You don't have to write anything here.  It is just a list for your reference when studying
\begin{itemize}
  \item isDigit
  \item isLetter
  \item isUpperCase
  \item isLowerCase
  \item toUpperCase
  \item toLowerCase
\end{itemize}

Summary of String methods you need to know. (You don't have to write anything here.  It is just a list for your reference when studying
\begin{itemize}
  \item length
  \item charAt
  \item format
  \item equals
  \item compareTo
  \item toUpperCase
  \item toLowerCase
  \item equals and equalsIgnoreCase
  \item compareTo and compareToIgnoreCase
\end{itemize}
\end{samepage}
 
\end{questions}

\end{document}
