% !TEX program = xelatex
% Notes template for CSC 254
% The document is based on the exams class at https://math.mit.edu/~psh/exam/examdoc.pdf
% 
% This document uses the csc254.sty file

\documentclass[letterpaper,11pt]{exam}
\RequirePackage{amssymb, amsfonts, amsmath, graphicx, latexsym, verbatim, xspace, setspace}

% By default LaTeX uses large margins.  This doesn't work well on exams; problems
% end up in the "middle" of the page, reducing the amount of space for students
% to work on them.
\usepackage[margin=1in]{geometry}
\usepackage{xcolor}

% Here's where you edit the Class, Exam, Date, etc.
\newcommand{\class}{CSC 254}
\newcommand{\term}{NOTES}
\newcommand{\examnum}{Week 10 Arrays}
\newcommand{\videoheading}[1]{\Large\textbf{\textit{#1}}}


% For an exam, single spacing is most appropriate
\singlespacing
% \onehalfspacing
% \doublespacing

% For an exam, we generally want to turn off paragraph indentation
\parindent 0ex
\begin{document} 

% These commands set up the running header on the top of the exam pages
\pagestyle{head}
\firstpageheader{}{}{}
\runningheader{\class}{\examnum\ - Page \thepage\ of \numpages}{\includegraphics[width=1in]{javant}}
\runningheadrule

\begin{flushright}
\begin{tabular}{p{2.8in} r l}
\textbf{\class} & \textbf{Name (Print):} & \makebox[2in]{\hrulefill}\\
\textbf{\term} &&\\
\textbf{\examnum} &&\\
\end{tabular}\\
\end{flushright}
\rule[1ex]{\textwidth}{.1pt}


This assignment contains \numpages\ pages (including this page) and
\numquestions\ questions.  Check to see if any pages are missing.\\
\videoheading{Video 10.010 About Arrays}
\begin{questions}

\question What is an array?
\vspace{1.5cm}
\question In Java, all elements in an array must be of the same type.
\begin{itemize}
  \item  True
  \item  False
  \item 
\end{itemize}
\question What does \texttt{\textbf{n}} usually represent

\question What is actually stored in the array variable?

\question Is it easy to make arrays bigger after they are declared in Java?

\videoheading{10.020 Args[]}
\question what is \texttt{\textbf{String[] args}}?
\vspace*{1cm}
\question How can you print the physical length of an array?  Does it have () after length?
\vspace{1cm}
\question Write a for loop to print the args[] array.
\vspace{4cm}
\question What would the above loop do if the args array is empty?
\question Assume that the args array has \textit{at least} 1 element.  What would the following statement print?
\begin{verbatim}
    System.out.println( args[args.length-1])
\end{verbatim}

\videoheading{10.030 Splitting Strings}
\question What does the \texttt{\textbf{split()}} method return?
\question What is the simplest type of regex?
\question Write the statement needed to split a spring on a # symbol and store the results in an array called \texttt{\textbf{things}}.
\vspace{1cm}
\question if a line is empty, then what is the length of the array that will be returned?  Why is this a potential problem?
\vspace{1cm}
\videoheading{10.040 Reading from a file}
\question What is a stream?  What are some sources for character streams?
\question What is a relative path?
\question Where should you put a file in Intellij in order to make it as simple as possible to read?

\question What extra steps do we have to do to read from a file that we did not need to do when reading from a keyboard?
\vspace{4cm}

\videoheading{10.050 Reading all the lines}
\question What data type is returned by the hasNext family of methods?
\question Write a loop that would read and echo print all of the Strings in a file (note that this is not using hasNextLine().  If you need to, look it up in the Scanner api.)
\vspace{4cm}

\videoheading{10.060 Declaring an array}
\question Declare an array that may hold up to 10 Strings.  Set a constant to 10 and be sure to also create a variable that represents the current number of elements in the array (or, the logical size of the array).
\vspace{2cm}

\videoheading{10.070 Reading a file into an array}
\question Write the \texttt{\textbf{while}} statement needed to read all the doubles from a file. The array may hold up to MAXSIZE doubles.
\question The following statement does two important things.  What are the two things it does?
\begin{verbatim}
    words[n++] = input.next();
\end{verbatim}

\question Write a complete method (function) that prints the items in an array.  Pass in the array and n to represent the logical size of the array.
\end{questions}


\end{document}
